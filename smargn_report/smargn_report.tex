%%%%%%%%%%%%%%%%%%%%%%%%%%%%%%%%%%%%%%%%%
% Structured General Purpose Assignment
% LaTeX Template
%
% This template has been downloaded from:
% http://www.latextemplates.com
%
% Original author:
% Ted Pavlic (http://www.tedpavlic.com)
%
% Note:
% The \lipsum[#] commands throughout this template generate dummy text
% to fill the template out. These commands should all be removed when
% writing assignment content.
%
%%%%%%%%%%%%%%%%%%%%%%%%%%%%%%%%%%%%%%%%%

%----------------------------------------------------------------------------------------
%	PACKAGES AND OTHER DOCUMENT CONFIGURATIONS
%----------------------------------------------------------------------------------------

\documentclass{article}
\usepackage[utf8]{inputenc}
\usepackage[T1]{fontenc}
\usepackage{lmodern} % load a font with all the characters

\usepackage{fancyhdr} % Required for custom headers
\usepackage{lastpage} % Required to determine the last page for the footer
\usepackage{extramarks} % Required for headers and footers
\usepackage{graphicx} % Required to insert images
\usepackage{lipsum} % Used for inserting dummy 'Lorem ipsum' text into the template
\usepackage{underscore}

% Margins
\topmargin=-0.45in
\evensidemargin=0in
\oddsidemargin=0in
\textwidth=6.5in
\textheight=9.0in
\headsep=0.25in

\linespread{1.1} % Line spacing

% Set up the header and footer
\pagestyle{fancy}
\lhead{\hmwkAuthorName} % Top left header
\chead{\hmwkTitle} % Top center header
\rhead{\firstxmark} % Top right header
\lfoot{\lastxmark} % Bottom left footer
\cfoot{} % Bottom center footer
\rfoot{Page\ \thepage\ of\ \pageref{LastPage}} % Bottom right footer
\renewcommand\headrulewidth{0.4pt} % Size of the header rule
\renewcommand\footrulewidth{0.4pt} % Size of the footer rule

\setlength\parindent{0pt} % Removes all indentation from paragraphs

%----------------------------------------------------------------------------------------
%	NAME AND CLASS SECTION
%----------------------------------------------------------------------------------------

\newcommand{\hmwkTitle}{Searching the space of word temporal profiles on Le Temps Newspaper } % Assignment title
\newcommand{\hmwkDueDate}{Monday,\ May 18,\ 2015} % Due date
\newcommand{\hmwkClassTime}{} % Class/lecture time
\newcommand{\hmwkClassInstructor}{} % Teacher/lecturer
\newcommand{\hmwkAuthorName}{} % Your name

%----------------------------------------------------------------------------------------
%	TITLE PAGE
%----------------------------------------------------------------------------------------

\title{
\vspace{2in}
\textmd{\textbf{\hmwkTitle}}\\
\normalsize\vspace{0.1in}\small{Due\ on\ \hmwkDueDate}\\
\vspace{3in}
}

\author{Sidney Bovet \and Valentin Rutz \and Zhivka Gucevska  \and Mathieu Monney\and Florian Junker\and John Gaspoz\and Joanna Salathé\and Ana Manasovska\and Fabien Jolidon}
\date{\today} % Insert date here if you want it to appear below your name

%----------------------------------------------------------------------------------------

\begin{document}

\maketitle

%----------------------------------------------------------------------------------------
%	TABLE OF CONTENTS
%----------------------------------------------------------------------------------------

%\setcounter{tocdepth}{1} % Uncomment this line if you don't want subsections listed in the ToC

\newpage
\tableofcontents
\newpage


\section{Motivation}
Studies based on the visualization and analyses of temporal profile of word using a so-called “n-gram” approach have been popular in the last years. However, most of the studies so far discuss the case of remarkable words, most of the time found thanks to the researcher’s intuitions for finding “Interesting” curves using the n-gram viewer.\\

The motivation of this project is to invert the problem and to propose a simple tool for researchers to automatically explore the space of temporal curves in search for words. As researchers often analyse how words appear and relate to historical events, our goal was to provide researchers a way to extract words having the same temporal profile. These results would help the researchers to analyze the use of a given word and the events that they relate to. \\

In order to classify the temporal profiles various metric were implemented providing the user different ways to retrieve interesting information. \\

In section 2, we will explain how the data generation was made. From parsing the OCR'd articles to a cleaned and formatted data. In Section 3 we detail how the application is structured and its general pipeline. Section 4 describes in details the metrics implemented as well as their strengths and weaknesses. Section 5 and 6 describe how we proceeded to optimize the metric's parameter and how the metrics perform against some test cases. Finally, section 7 provides hints about future works that could fit this project.\\

Our analysis is based on Le Temps Newspaper Corpus which is a database of 4 million articles covering a period of 200 years and composed of digitized facsimiles of "Le Journal de Genève" and "La Gazette de Lausanne" wherein each article has been OCR'd.



\section{Data Generation}
Data generation is a critical component as messy data will impact the accuracy of the metrics and the pertinence of its outputs. Hence, great care was taken in the data generation phase in order to make the data as useful as possible for the application. With the use of various statistics we were able to filter many garbage data.

\subsection{Parsing}

\subsection{1gram Generation}
To ease the development of the metrics we defined that for each word we would have a list of 159 Integer entries containing the number of occurrences for this word for each year in the range of our database ( from year 1840 to 1998).\\

In order to compute this, two map reduce functions were created. The parsing phase had given a list of files, one file for a year, containing every word used in that year. The first mapreduce would process each of these files and output, in a file, each word associated with the year and occurrences for this year : [year\_word occurrences].\\

The second mapreduce would take, as input, the output of the first one and generate the final format [word occurence\_year1, occurence\_year2, ...]. If not occurence of a word was found for a year a 0 value would be assigned to this specific year.

\subsection{1gram Cleaning}

\section{Architecture}

\section{Metrics}

\subsubsection{Curve Comparison}

\subsubsection{Peak Detection}

\subsubsection{Dynamic Time Warping}

\section{Data Tuning}

\section{Experimental results}
\subsection{Metric Benchmark}

\section{Conclusion and Future work}

As observed in the previous section, with the use of this application we were able to retrieve interesting information. For example, by testing the word "guerre" we detected various semantically similar words as output. And by looking at their temporal profiles we can analyse when these words were mostly used and observe that their pic of occurrences would match the historic event of the first and second war. \\

This a simple test-case but it shows that, giving a word, the user can retrieve useful information. Of course not all words would give such interesting results and, as we saw previously, not every metric performs well for any words.



\end{document}



